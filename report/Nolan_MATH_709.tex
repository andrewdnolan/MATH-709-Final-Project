% SIAM Article Template
\documentclass[review,onefignum,onetabnum]{siamart171218}

% Information that is shared between the article and the supplement
% (title and author information, macros, packages, etc.) goes into
% ex_shared.tex. If there is no supplement, this file can be included
% directly.

\input{ex_shared}

% Optional PDF information
\ifpdf
\hypersetup{
  pdftitle={Linear Algebra and Adaptive Mesh Refinement},
  pdfauthor={A. Nolan}
}
\fi

% The next statement enables references to information in the
% supplement. See the xr-hyperref package for details.

%% Use \myexternaldocument on Overleaf
\myexternaldocument{ex_supplement}

\begin{document}

\maketitle

% REQUIRED
\begin{abstract}
    An introductory description of finite volume methods for solving partial differential equations is discussed below. Additionally, apadtive mesh refinment is disscussed in some detail. 
\end{abstract}

% REQUIRED
% \begin{keywords}
%   example, \LaTeX
% \end{keywords}

% REQUIRED
% \begin{AMS}
%   68Q25, 68R10, 68U05
% \end{AMS}

\section{Introduction}
Partial differential equations arise across all scientific fields. Much of the most interesting scientific information lies within non-linear partial differential equations for which analytical solutions are not available or impractical. This motivates our interest in solutions to discretzied PDEs since they provide numerical approximations to the equations in question. The most basic, and starting point for numerical solutions to PDEs is the finite difference method which numerically approximates derivatives using the Taylor expansion, reducing ODEs and PDEs to algebraic statements easily solved. 


We begin with an introduction into finite volume solutions to discretized partial differential equations. 

% The outline is not required, but we show an example here.
The paper is organized as follows. Our main results are in
\cref{sec:main}, our new algorithm is in \cref{sec:alg}, experimental
results are in \cref{sec:experiments}, and the conclusions follow in
\cref{sec:conclusions}.


motivation for higher order slope limiters is that incroporate information about higher order deriavtes so less spurious oscialltions at steep gradients, i.e. schock waves 
\section{Methods}
\label{sec:methods}

\subsection{Finite Volume}
We begin by deriving the finite volume method via conservation laws from one dimensional advection. Our derivation closely follows that of \cite{comp_seis} which in turn closely follows that of \cite{leveque_2002}. Let us consider a fluid flow such a dye tracer in a river or a supraglacial stream, with a positive advection speed ($a$), that is the fluid is flowing from left to right. The total mass of the quantity in question (i.e. concentration of the tracer) in a given unit volume is 
\begin{equation}
    \int_{x_l}^{x_r} q(x,t) \: dx
\end{equation}
Given that we are working with the one dimensional advection equation the only changes in mass within our given unit of volume with be through flux either through the left or right cell boundaries. Therefore
\begin{equation}
    \frac{\partial}{\partial t} \int_{x_l}^{x_r} q(x,t) \: dx = F_l(t) - F_r(t)
\end{equation}
where $F_i(t)$ are mass fluxes with units (). We can rewrite the fluxes $F_i$ as function of the values of $q(x,t)$ as 
\begin{equation}
    F \xrightarrow{} f(q(x,t)) 
\end{equation}
and therefore we can write our the chnage in unit volume with time as 
\begin{equation}
    \frac{\partial}{\partial t} \int_{x_l}^{x_r} q(x,t) \: dx = f(q(x_l,t))  - f(q(x_r ,t)) 
    \label{eq:intergralform}
\end{equation}
which represents the integral form of one dimensional hyperbolic conservation law. This integral form of our conservation law is incredibly usefully for numerical studies since the wave equation and other hyperbolic PDEs are prone to discontinuities in the solution between the boundaries between heterogenous media or as a wave breaks as control by the batherymetry over which the wave flows. The these discontinuities or shocks the governing PDE may not hold or be impossible to solve and result in large errors in other nuemrical approximations. But since the finit volume method was consived with these exact situations in mind. 

To further simplify the perivous equation we use the definition of a defint intergal and find 
\begin{equation}
     \frac{\partial}{\partial t} \int_{x_l}^{x_r} q(x,t) \: dx= - \int_{x_l}^{x_r} f(q(x,t))\: dx
\end{equation}
which can be written as 
\begin{equation}
    \int_{x_l}^{x_r} \left[ \frac{\partial}{\partial t} q(x,t) -  f(q(x,t)) \right] \: dx = 0
\end{equation}
Which leads to the well know one dimensional advection equation
\begin{equation}
    \frac{\partial}{\partial t} q - f(q) = 0 
\end{equation}
The equation above has a constant scalar velocity of 1. If for example we have a non constant cofficent such that 
\begin{equation}
    \frac{\partial}{\partial t} q - A(x,t) \: f(q) = 0 
\end{equation}
\subsubsection{Derivation}

\subsubsection{First and Second Order Approximations}

We begin by deriving a first order numerical solution to the constant coefficent eqaution 
\begin{equation}
     \frac{\partial q}{\partial t} - u\: \frac{\partial q}{\partial x} = 0 
\end{equation}
on a numerical grid where the $i$-th grid cell, $\mathcal{C}_i$ has cell interfaces $x_{i-1/2}, x_{i+1/2}$. We can approximate the value of the solution field $q(x,t)$ by the average quantity $Q_i^n$ within a given grid cell as 
\begin{equation}
    Q_i^n  = \frac{1}{dx} \int_\mathcal{C} q(x,t)\: dx \approx \frac{1}{\Delta x} \int_\mathcal{C} q(x,t) \: dx
\end{equation}
were the subscript denotes the grid cell and the superscript denotes the time integration step and $\Delta x$ is the size of the gird cell. We use the above definition of average quantity along with the integral form of our conservation law (Equation. \cref{eq:intergralform}) to derive and explicit algorithm. By integrating Equation  \cref{eq:intergralform} from $t_n$ to $t_{n+1}$ we then get 
\begin{equation}
    \int_{x_l}^{x_r} q(x,t_{n+1}) \: dx - \int_{x_l}^{x_r} q(x,t_n) \: dx = \int_{t_n}^{t_{n+1}} (q(x_l,t)) \: dt  - \int_{t_n}^{t_{n+1}} f(q(x_r ,t)) \: dt
\end{equation}
In order for our intergrated equation to match the cell avergared form derived above we devide all terms by $\Delta x$ and then solve for $q(x,t_{n+1})$ and end up with 
\begin{align}
    \frac{1}{\Delta x} \int_{x_l}^{x_r} q(x,t_{n+1}) dx  = \frac{1}{\Delta x} \int_{x_l}^{x_r} q(x,t_n) dx \\
    - \frac{1}{\Delta x} \left[ \int_{t_n}^{t_{n+1}} f(q(x_r ,t)) dt - \int_{t_n}^{t_{n+1}} f(q(x_l,t)) dt \right] \notag
\end{align}
We note that our first two terms are the cell averaged solution field then second two terms can be written as 
\begin{equation}
    F_{i \pm 1/2} \approx \frac{1}{\Delta t} \int_n^{n+1} f(q(x_{i\pm 1/2}, t))
\end{equation}
which allows us to rewrite the equation above as 
\begin{align}
    Q_i^{n+1}  = Q_i^{n} - \frac{\Delta t}{\Delta x} \left[ F^n_{i + 1/2} - F^n_{i - 1/2} \right]
\end{align}
Under the assumption that $F_{i \pm 1/2}$ can be approximated by just using the flux at $\{Q_{i-1}, Q_i\}$ we can state the numerical flux as
\begin{align}
    \mathcal{F}_{i\pm 1/2}^n = \mathcal{F} \left( Q, Q\right)
\end{align}
Which leads to the numerical method of the form 
\begin{equation}
     Q_i^{n+1}  = Q_i^{n} - \frac{\Delta t}{\Delta x} \left[ \mathcal{F}^n_{i + 1/2} - \mathcal{F}^n_{i - 1/2} \right]
\end{equation}
How we choose to numerical approximate that flux controls the order of the numerical method. 
\subsubsection{Higher Order Approximations}


\subsection{Adaptive Mesh Refinement}

\section{Results}

\subsection{1-D Elastic Wave Equation}

\subsection{GeoClaw}

\section*{Acknowledgments}
We would like to acknowledge the assistance of volunteers in putting
together this example manuscript and supplement.

\bibliographystyle{siamplain}
\bibliography{references}
\end{document}
